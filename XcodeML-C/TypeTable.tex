\section{ {\tt typeTable} element}

The {\tt typeTable} element is used to define the data type information for the entire file. The element is one of the data type definition elements. Data type definition elements comprise the following elements:

\begin{itemize}
\item {\tt pointerType} element,
\item {\tt functionType} element,
\item {\tt arrayType} element,
\item {\tt structType} element,
\item {\tt unionType} element, 
\item {\tt enumType} element, and
\item {\tt basicType} element.
\end{itemize}


\subsection{Data type names}

Data in a program are differentiated using data type names. These names are one of the following two type names:

\begin{itemize}
\item Basic data type names, which can be further split into those
      that correspond to C language basic data types:
      \vspace{2mm}

      '{\tt void}', '{\tt char}', '{\tt short}', '{\tt int}' , '{\tt long}', '{\tt long\_long}', '{\tt unsigned\_char}', '{\tt unsigned\_short}', '{\tt unsigned}', '{\tt unsigned\_long}', '{\tt unsigned\_long\_long}', '{\tt float}', '{\tt double}', '{\tt long\_double}', '{\tt wchar\_t}', '{\tt bool}' ('{\tt \_Bool} type)
      \vspace{2mm}

      that correspond to {\tt \_Complex} and {\tt \_Imaginary} types:
      \vspace{2mm}

      '{\tt float\_complex}', '{\tt double\_complex}', '{\tt long\_double\_complex}', '{\tt float\_imaginary}', '{\tt double\_imaginary}', '{\tt long\_double\_imaginary}'
      \vspace{2mm}

      that correspond to types built into the GCC:
      \vspace{2mm}

      '{\tt \_\_builtin\_va\_arg}'
      \vspace{2mm}

\item Derived data type names - arbitrary alphanumeric strings that are not part of the above mentioned basic data type names
      Derived data type names must be unique within a program.
\end{itemize}


\subsection{ {\tt basicType} element}

The {\tt basicType} defines the basic data type for C and C99. The element has the following attributes:

\begin{itemize}
\item {\tt type} and
\item {\tt name}
\end{itemize}

\subsubsection*{Example}
\vspace{1mm}

\begin{CExample}
struct {int x; int y;} s;
struct s const * volatile p;
\end{CExample}

is converted to the following XcodeML
\vspace{2mm}

\begin{XcodeMLExample}
  <structType type="S0">
    ...
  </structType>
  <basicType type="B0" is_const="1" name="S0"/>
  <pointerType type="P0" is_volatile="1" ref="B0"/>
\end{XcodeMLExample}


\subsection{ {\tt pointerType} element}

The {\tt pointerType} element defines a pointer data type. The element has the following attributes:

\begin{itemize}
\item {\tt type} - the derived data type name for the pointer type and
\item {\tt ref} - the data type name used for referencing the pointer type and data type.
\end{itemize}

The {\tt pointerType} element does not possess any other elements.

\subsubsection*{Example}

The following data type definition corresponds to "{\tt int *}".
\vspace{2mm}

\begin{XcodeMLExample}
  <pointerType type="P0123" ref="int" />
\end{XcodeMLExample}


\subsection{ {\tt functionType} element}

The {\tt functionType} element defines a function data type.

\begin{itemize}
\item {\tt type} - the derived data type name for the function type.
\item {\tt return\_type} - name of the returned data type for the function type.
\item {\tt is\_inline} - specifies whether the function type is an inline type, using {\tt 0} or {\tt 1} ({\tt false} or {\tt true}).
\end{itemize}

If there is a prototype declaration, it includes the {\tt param} elements corresponding to the argument elements.

\subsubsection*{Example}

For "{\tt double foo(int a, int b)}", the following corresponds to the "{\tt foo}" data type.
\vspace{2mm}

\begin{XcodeMLExample}
  <functionType type="F0457" return_type="double">
  <params>
      <name type="int">a</name>
      <name type="int">b</name>
    </params>
  </fucntionType>
\end{XcodeMLExample}


\subsection{ {\tt arrayType} element}

The {\tt arrayType} element defines an array data type. The {\tt arrayType} element has the following attributes:

\begin{itemize}
\item {\tt type} - the derived data type name for the array type,
\item {\tt element\_type} - specifies the identifier for the array element data type,
\item {\tt array\_size} - specifies the size of the array (number of elements). Omitting {\tt array\_size} and its subelement {\tt arraySize} corresponds to not specifying the size. Attributes for {\tt array\_size} and its subelement {\tt arraySize} cannot be specified simultaneously. 
\item {\tt is\_const}, {\tt is\_volatile}, {\tt is\_restrict}, {\tt is\_static} - indicates whether these attributes specify each of the {\tt const}, {\tt volatile}, {\tt restrict}, and {\tt static} modifiers of the array size. The value for each is {\tt 0} or {\tt 1} ({\tt false} or {\tt true}).
\end{itemize}

The element has the following subelement:

\begin{itemize}
\item {\tt arraySize} - Expression specifying the size of the array (number of elements). The element has an expression subelement.

      When the size cannot be specified using a number, specify a variable-length array. When an {\tt arrayType} element has an {\tt arraySize} subelement, the {\tt array\_size} attribute value is "{\tt *}". 
\end{itemize}

\subsubsection*{Example}

In "{\tt int a[10]}" the type\_entry corresponding to "{\tt a}" is as follows.
\vspace{2mm}

\begin{XcodeMLExample}
<arrayType type="A011" element_type="int" array_size="10"/>
\end{XcodeMLExample}


\subsection{ {\tt structType} and {\tt unionType} elements}

A struct (structure) data type is defined using the {\tt structType} element. The structType element has the following attribute:

\begin{itemize}
\item {\tt type} - the derived data type name for the array type.
\end{itemize}

A union data type is defined using the {\tt unionType} element. The {\tt unionType} and {\tt structType} elements have the same attributes and elements.
The {\tt structType} and {\tt unionType} elements have symbolic elements that contain member identifier information. When structure and union tag names exist, they are defined in a symbol table that corresponds to the scope.
The member bit field is described in the {\tt bit\_field} attribute of the {\tt id} element, or in its subelement's {\tt bitField} tag. The {\tt bitField} tag has an expression subelement. Furthermore, the {\tt id} element, which possesses the {\tt bitField} tag, has a {\tt bit\_field} attribute, whose value is "{\tt *}". 

\subsubsection*{Example}

In "{\tt struct {int x; int y : 8; int z : sizeof(int); } S;}" the {\tt structType} element corresponding to {\tt S} is as follows:

\begin{XcodeMLExample}
  <stuctType type="S6e89">
    <symbols>
      <id type="int">
        <name>x</name>
      </id>
      <id type="int" bit_field="8">
        <name>y</name>
      </id>
      <id type="int" bit_field="*">
        <name>z</name>
        <bitField>
          <sizeOfExpr>
            <typeName ref="int"/>
          </sizeOfExpr>
        </bitField>
      </id>
     </symbols>
  </structType>
\end{XcodeMLExample}


\subsection{ {\tt enumType} element}

The {\tt enumType} element defines the {\tt enum} type. The type element specifies the member identifiers.
This element has the following subelement:

\begin{itemize}
\item {\tt symbols} - defines the member identifiers. Initial values for the members are specified by the {\tt id} - {\tt value} element.
\end{itemize}

Member identifiers are defined in the {\tt moe} class of a symbol table that corresponds to the scope.  When {\tt enum} tag names exist, they are defined in a symbol table that corresponds to the scope.

\subsubsection*{Example}

In "{\tt enum { e1, e2, e3 = 10 } ee; }" the {\tt enumType} element corresponding to "{\tt ee}" is as follows.
\vspace{2mm}

\begin{XcodeMLExample}
  <enumType name="E0">
    <symbols> 
      <id>
        <name>e1</name>
      </id>
      <id>
        <name>e2</name>
      </id>
      <id>
        <name>e3</name>
        <value><intConstant>10</intConstant></value>
      </id>
    </symbols>
  </enumType>
\end{XcodeMLExample}


\subsection{Optional attributes for data type definition elements}

The following are attributes for data type definition elements (these attributes can be omitted).

\begin{itemize}
\item {\tt is\_const} - specifies whether or not the variable defined for the data type is constant using {\tt 0} or {\tt 1} ({\tt false} or {\tt true});
\item {\tt is\_volatile} - specifies whether or not the variable defined for the data type is volatile using {\tt 0} or {\tt 1} ({\tt false} or {\tt true});
\item {\tt is\_restrict} - specifies whether or not the variable defined for the data type is restricted using {\tt 0} or {\tt 1} ({\tt false} or {\tt true});
\end{itemize}


