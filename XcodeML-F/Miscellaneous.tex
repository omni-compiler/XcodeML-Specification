\section{Miscellaneous}

\subsection{Intrinsic procesures}

Intrinsic procesures are treated as external functions or subroutines
without {\tt extern} declaration.


\subsection{Treatment of comemts and pragmas}

There is no element which represents comments. However, the pragma elements for the directives
of OpenMP 3.0 are kept by the {\tt FpragmaStatement} elements.


\subsection{Canonicalization}

In this specification of XcodeML/Fortran, codes are assumed to be altered in the actual
conversion proccess by the frontend within the range where the meanings of the codes are not changed.
The alterations are made as follows:

\begin{itemize}
\item As for the {\tt implicit} statement, declares all the variables explicitly and adds the {\tt implicit none} statement.
\item The {\tt parameter} statement, the {\tt dimension} statement, the {\tt allocatable} statement,
the {\tt pointer} statement, the {\tt target} statement, the {\tt intent} statement and the {\tt save} statement
saving a variable are unified into a type declaration of the variable.
\item The {\tt parameter} variable initialized by a constant is replaced by the constatnt.
\item The {\tt statement function} is inlined.
\item The unary operator '{\tt +}' is omitted.
\item Any comments other than the directives of OpenMP 3.0 are removed.
\end{itemize}

Accordingly, some Fortran 90 syntax elements such as the {\tt statement function} have no corresponding elements in XcodeML/Fortran.

\subsection{Restrictions}

This specification of XcodeML/Fortran does not support the following Fortran 90 syntaxes because they do not appear in both of NBP and SPEC benchmarks.

\begin{itemize}
\item The {\tt block data} statement, the {\tt end block data} statement.
\item The {\tt optional} statement within the {\tt module} declaration.
\item The binary, octal and hexadecimal integer expression by the BOZ literal.
\end{itemize}

On the other hand, the following 6 items from the obsolescent features in Fortran 90 are not supported.

\begin{itemize}
\item The {\tt do} variable and the expressions of type default real or double precision real, and the shared do-loop terminataion.
\item The {\tt assign} statement, the {\tt assigned go to} statement and the {\tt assigned format}.
\item Branching to an {\tt end if} statement.
\item The alternate {\tt return} and the {\tt alternate return} statement.
\item The {\tt pause} statement.
\item The {\tt H edit descriptor}.
\end{itemize}


