\section{Introduction}


This document describes the specification of XcodeML for Fortran 90, which is defined based on XcodeML originaly defined as an intermediate code of the computer languages, mainly for C. To clearly distinguish both specifications, we call this specification XcodeML/Fortran. The version of XcodeML on which this is based is 0.8J.


XcodeML/Fortran has the following functions and characteristics:

\begin{itemize}
\item Preserves information that can be used to reconstruct a Fortran 90 program,
\item Can represent the type information of the Fortran 90 programming language,
\item Has syntax elements necessary for a variety of transformations, and
\item Has a human-readable format (XML).
\end{itemize}

XcodeML and XcodeML/Fortran are primarily designed to be convenient to use as an intermidiate code in the source code to source code converter systems(hereafter called translators). The program which converts code written in a computer language to XcodeML or XcodeML/Fortran is called the frontend program(hereinafter called frontend), the one which converts XcodeML or XcodeML/Fortran to code written in the computer language is called the backend program(hereinafter called backend) or the decompiler. In addition, there are analisys/converter programs which input/output intermediate code to statically analyze or parallelize a program within the source code tralslator systems.
